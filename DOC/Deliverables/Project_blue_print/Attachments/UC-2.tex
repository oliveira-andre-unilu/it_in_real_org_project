\subsection{Use Case: Employee Shift Punch-in (Start
Shift)}\label{use-case-employee-shift-punch-in-start-shift}

\textbf{Use Case ID:} UC-2\\
\textbf{Version:} 1.0\\
\textbf{Created:} 19/10/2025\\
\textbf{Last Updated:} 19/10/2025\\
\textbf{Priority:} \texttt{High} \\
\textbf{Status:} \texttt{Draft}\\
\textbf{Related US:} US-2

\textbf{Primary Actor:} - Employee (Application User)

\textbf{Secondary Actors:} - Employer - Team Leader - Management

\textbf{Stakeholders:} - Company

\textbf{Brief Description:} This use case describes how an employee
starts logging their working hours by using a ``Start Shift'' function
in the mobile application.\\
The process records the current timestamp and the GPS location (Car GPS
in the future) of the user. This data is sent to the central server
after the start of a shift, as soon as there is a stable internet
connection (in case of connectivity issue), or after marking the end of
the shift.\\
This feature ensures accurate time and location tracking and minimizes
manual input or reporting errors.

\textbf{Trigger:} The user opens the mobile application and in no more
than 3 taps, presses ``Start Shift'' button to begin recording their
working hours.

\textbf{Preconditions:} - The user must be already logged into their
account in the application. - The application must have the appropriate
permissions to access time and location data of the device and time
servers (critical to prevent time manipulation). - The server or local
storage must be available to record the shift start data.

\textbf{Postconditions:} - The system successfully records the shift
start timestamp and GPS coordinates. - The data is stored on the central
(or queued locally for later synchronization). - The applications
displays a confirmation that the shift has started.

\subsubsection{Main Success Scenario}\label{main-success-scenario}

\begin{longtable}[]{@{}
  >{\raggedright\arraybackslash}p{(\linewidth - 4\tabcolsep) * \real{0.1622}}
  >{\raggedright\arraybackslash}p{(\linewidth - 4\tabcolsep) * \real{0.3784}}
  >{\raggedright\arraybackslash}p{(\linewidth - 4\tabcolsep) * \real{0.4595}}@{}}
\toprule\noalign{}
\begin{minipage}[b]{\linewidth}\raggedright
Step
\end{minipage} & \begin{minipage}[b]{\linewidth}\raggedright
Actor Action
\end{minipage} & \begin{minipage}[b]{\linewidth}\raggedright
System Response
\end{minipage} \\
\midrule\noalign{}
\endhead
\bottomrule\noalign{}
\endlastfoot
1 & User opens the mobile application. & The application loads the
home/dashboard screen. \\
2 (Optional) & User navigates to shifts of a specific site/location. &
The application loads a screen with a ``Start Shift'' button for the
specific site/location. \\
3 & User taps the ``Start Shift'' button. & The system validates that no
active shift exists for the user. \\
4 & & The system records the current timestamp and GPS coordinates. \\
5 & & The system stores the data in the local database and/or sends it
to the central server. \\
6 & & The application displays a confirmation message (example: ``Shift
started at 08:32''). \\
7 & User taps ``Ok'' on the confirmation message. & The application
updates the UI to show the ongoing shift status (example: ``Shift
started at 08:32 (Time worked: 02:45)''). \\
\end{longtable}

\subsubsection{Extensions (Alternative
Flows)}\label{extensions-alternative-flows}

\textbf{2a. No Internet Connection}

\begin{itemize}
  \item \textbf{Condition:} The application cannot reach the central server.
  \item \textbf{Action:} The system stores the shift start data locally.
  \item \textbf{Result:} Data will be automatically synced once the device reconnects to the internet. 
\end{itemize}


\textbf{3a. GPS unavailable}
\begin{itemize}
  \item \textbf{Condition:} The phone's GPS is disabled or unavailable.
  \item \textbf{Action:} The system records the timestamp only and logs a ``Location Missing'' flag or return location as ``null''.
  \item \textbf{Result:} The shift still starts successfully, butwith a warning message.
\end{itemize}


\textbf{4a. Duplicate Shift Attempt}\\
\emph{NOTE: This should probably never occur but safety fallbacks should
still be implemented!}
\begin{itemize}
  \item \textbf{Condition:} User tries to start another shift while one is already active.
  \item \textbf{Action:} The system prevents the action and shows an error message (example: ``Shift already active'').
  \item \textbf{Result:} The application handles the issue in an automated manner without creating duplicate entries.
\end{itemize}

\subsubsection{Special Requirements}\label{special-requirements}

\textbf{Performance:}\\ - The system should record the shift start within
\textbf{about 2 seconds} of user action. - The ``Start Shift'' button
should be accessible within \textbf{3 taps or fewer} from the launch of
the application.

\textbf{Security:}\\ - User authentication via secure credentials or
company SSO. - Shift data must be encrypted during transmission.

\textbf{Reliability:}\\ - System should handle temporary network outages
by caching data locally. - The application should automatically retry
synchronization when connection is restored.

\textbf{Technical Constraints:}\\ - Mobile app must be compatible with
Android at launch (iOS can be released at a later stage). - Integration
of GPS should keep vehicle tracking system in mind.

\subsubsection{Open Issues}\label{open-issues}

\begin{itemize}
\tightlist
\item
  Should location tracking be mandatory or optional due to privacy
  concerns?
\item
  How will unsuccessful shift start be handled when the user performs
  the actions correctly but there is a failure in the system? (Avoid
  creating issues to employees)
\end{itemize}
