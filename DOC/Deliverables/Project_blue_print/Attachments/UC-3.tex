\subsection{Use Case: Employee Shift Punch-in (Start
Shift)}\label{use-case-employee-shift-punch-in-start-shift}

\textbf{Use Case ID:} UC-3\\
\textbf{Version:} 1.0\\
\textbf{Created:} 19/10/2025\\
\textbf{Last Updated:} 19/10/2025\\
\textbf{Priority:} \texttt{High} \\
\textbf{Status:} \texttt{Draft} \\
\textbf{Related US:} US-3

\textbf{Primary Actor:} - Employee (Application User)

\textbf{Secondary Actors:} - Employer - Team Leader - Management

\textbf{Stakeholders:} - Company

\textbf{Brief Description:} This use case describes how an employee
stops logging their working hours by using a ``End Shift'' function in
the mobile application.\\
The process records the current timestamp and the GPS location (Car GPS
in the future) of the user. This data is sent to the central server
after the end of a shift or as soon as there is a stable internet
connection (in case of connectivity issue).\\
This feature ensures accurate time and location tracking and minimizes
manual input or reporting errors.

\textbf{Trigger:} The user opens the mobile application and in no more
than 3 taps, presses ``End Shift'' button to stop recording their
working hours.

\textbf{Preconditions:} - The user must be already logged into their
account in the application. - The application must have the appropriate
permissions to access time and location data of the device and time
servers (critical to prevent time manipulation). - The server or local
storage must be available to record the shift end data.

\textbf{Postconditions:} - The system successfully records the shift end
timestamp and GPS coordinates. - The data is stored on the central (or
queued locally for later synchronization). - The applications displays a
confirmation that the shift has ended. - The total hours worked are
calculated (slightly rounded up) and displayed to the user. - The shift
entry is marked as completed in the database.

\subsubsection{Main Success Scenario}\label{main-success-scenario}

\begin{longtable}[]{@{}
  >{\raggedright\arraybackslash}p{(\linewidth - 4\tabcolsep) * \real{0.1622}}
  >{\raggedright\arraybackslash}p{(\linewidth - 4\tabcolsep) * \real{0.3784}}
  >{\raggedright\arraybackslash}p{(\linewidth - 4\tabcolsep) * \real{0.4595}}@{}}
\toprule\noalign{}
\begin{minipage}[b]{\linewidth}\raggedright
Step
\end{minipage} & \begin{minipage}[b]{\linewidth}\raggedright
Actor Action
\end{minipage} & \begin{minipage}[b]{\linewidth}\raggedright
System Response
\end{minipage} \\
\midrule\noalign{}
\endhead
\bottomrule\noalign{}
\endlastfoot
1 & User opens the mobile application. & The application loads the
home/dashboard screen. \\
2 (Optional) & User navigates to shifts of a specific site/location. &
The application loads a screen with a ``End Shift'' button for the
specific site/location. \\
3 & User taps the ``End Shift'' button. & The system validates that an
active shift exists for the user. \\
4 & & The system records the current timestamp and GPS coordinates. \\
5 & & The system stores the data in the local database and/or sends it
to the central server. \\
6 & & The application displays a confirmation message (example: ``Shift
ended at 15:28. Total hours worked: 7 h 56 min''). \\
7 & User taps ``Ok'' on the confirmation message. & The application
updates the UI to show ``Start Shift'' once again. \\
\end{longtable}

\subsubsection{Extensions (Alternative
Flows)}\label{extensions-alternative-flows}

\textbf{2a. No Internet Connection}\\ - \textbf{Condition:} The
application cannot reach the central server.\\ - \textbf{Action:} The
system stores the shift end data locally.\\ - \textbf{Result:} Data will
be automatically synced once the device reconnects to the internet.

\textbf{3a. User Cancels Confirmation}\\ - \textbf{Condition:} User
presses ``Cancel'' in the confirmation dialog.\\ - \textbf{Action:} The
system aborts the operation.\\ - \textbf{Result:} The shift remains active
and the recorded shift end data is wiped.

\textbf{4a. GPS unavailable}\\ - \textbf{Condition:} The phone's GPS is
disabled or unavailable.\\ - \textbf{Action:} The system records the
timestamp only and logs a ``Location Missing'' flag or return location
as ``null''.\\ - \textbf{Result:} The shift still ends successfully, but
with a warning message.

\textbf{5a. No Active Shift Found}\\
\emph{NOTE: This should probably never occur but safety fallbacks should
still be implemented!}\\ - \textbf{Condition:} User attempts to end a
shift when none is active.\\ - \textbf{Action:} The application displays
an error (example: ``No active shift to end'').\\ - \textbf{Result:}
Action terminates without recording any data.

\subsubsection{Special Requirements}\label{special-requirements}

\textbf{Performance:}\\ - End-of-shift confirmation and processing should
complete \textbf{within 3 seconds}.\\ - The ``End Shift'' button should be
accessible within \textbf{3 taps or fewer} from the launch of the
application.

\textbf{Security:}\\ - User authentication via secure credentials or
company SSO. - Shift data must be encrypted during transmission.

\textbf{Reliability:}\\ - System should handle temporary network outages
by caching data locally.\\ - The application should automatically retry
synchronization when connection is restored.

\textbf{Technical Constraints:}\\ - Mobile app must be compatible with
Android at launch (iOS can be released at a later stage).\\ - Integration
of GPS should keep vehicle tracking system in mind.

\subsubsection{Open Issues}\label{open-issues}

\begin{itemize}
\tightlist
\item
  Should location tracking be mandatory or optional due to privacy
  concerns?
\item
  How will unsuccessful shift end be handled when the user performs the
  actions correctly but there is a failure in the system? (Suggestion:
  Have a fixed amount of hours for the shift of each employee (usually 8
  hours). Once 8 hours have passed, end shift automatically.)
\item
  Should the confirmation dialog include additional notes or comments
  (e.g., ``Reason for early departure'')?
\end{itemize}
